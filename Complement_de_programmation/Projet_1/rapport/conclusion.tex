% !TEX root = ./main.tex
%%%%%%%%%%%%%%%%%%%%%%%%%%%%%%%%%%%%%%%%%%%%%%%%%%%%%%%%%%%%%%%%%%%%%%%%%%%%%%%%%%%%%%%%%%
% Rédigez ici la conclusion de votre rapport.                                            %
%%%%%%%%%%%%%%%%%%%%%%%%%%%%%%%%%%%%%%%%%%%%%%%%%%%%%%%%%%%%%%%%%%%%%%%%%%%%%%%%%%%%%%%%%%
\section{Conclusion}\label{conclusion}
%%%%%%%%%%%%%%%%%%%%%
Pour conclure, à travers notre rapport, nous avons établi une approche
constructive pour une fonction calculant le plus long
sous-tableau qui est à la fois préfixe et suffixe d'un tableau donné. 
Nous avons défini le problème, formalisé le problème, analysé, découpé en 
sous-problèmes et avons fait une approche constructive sur la base de nos invariants.

\vspace{0.4cm}
Nous avons également calculé la complexité de notre algorithme, qui est $O(n^2)$.
Notre code peut désormais être utilisé pour résoudre le problème de manière efficace et
fiable.