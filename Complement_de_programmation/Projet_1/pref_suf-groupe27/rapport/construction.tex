% !TEX root = ./main.tex
%%%%%%%%%%%%%%%%%%%%%%%%%%%%%%%%%%%%%%%%%%%%%%%%%%%%%%%%%%%%%%%%%%%%%%%%%%%%%%%%%%%%%%%%%%
% Dans cette section, il est demandé d'appliquer l'approche constructive pour la         %
% construction de votre code.                                                            %
%                                                                                        %
% Pour chaque Sous-Problème (une sous-section/SP):                                       %
% - {Pré} INIT {INV}                                                                     %
% - déterminer le Critère d'Arrêt (et donc le Gardien de Boucle)                         %
% - {INV & B} ITER {INV}                                                                 %
% - {INV & !B} END {Post}                                                                %
% - Fonction de Terminaison (pensez à justifier sur base de l'Invariant Graphique)       %
% (une sous-sous-section/tiret)                                                          %
%%%%%%%%%%%%%%%%%%%%%%%%%%%%%%%%%%%%%%%%%%%%%%%%%%%%%%%%%%%%%%%%%%%%%%%%%%%%%%%%%%%%%%%%%%
\section{Approche Constructive}
%%%%%%%%%%%%%%%%%%%%%%%%%%%%%%%%

\subsection{Sous-problème 1}

Voici le code ainsi que la construction instruction par instruction pour le sous-problème 1.
\begin{lstlisting}[caption={Sous-problème 1}]
    (>\coms{\{Pré $\equiv N = N_{0} \land 0 \leq N$\}}<)
    int k = N - 1;
    (>\coms{\{Inv $\equiv N = N_{0} \land 0 \leq N \land k = N - 1 \land k \ne \text{est\_pref\_suff}(\text{*}T, i, N)$\}}<)
    while (k > 0) {
      (>\coms{\{Inv $\land B \equiv N = N_{0} \land 0 \leq N \land k \leq N - 1 \land k \ne \text{est\_pref\_suff}(\text{*}T, i, N) \land k > 0$\}}<)

      (>\coms{\{$N = N_{0} \land 0 \leq N \land 0 < k \leq N - 1 \land k \ne \text{est\_pref\_suff}(\text{*}T, i, N)$\}}<)
      // SP2
      if (i == k){
        (>\coms{\{$N = N_{0} \land 0 \leq N \land 0 < k \leq N - 1 \land k = \text{est\_pref\_suff}(\text{*}T, i, N)$\}}<)
        return k;
      }
      else{
        i = 0;
      }
      k --;
      (>\coms{\{Inv $\land B \equiv N = N_{0} \land 0 \leq N \land k \leq N - 1 \land k \ne \text{est\_pref\_suff}(\text{*}T, i, N)$\}}<)
    }
    (>\coms{\{Inv $\land \lnot B \equiv T = T_{0} \land N \geq 0 \land k = 0$\}}<)
\end{lstlisting}


\subsection{Sous-problème 2}

Voici les instructions ainsi que la construction instruction par instruction pour le sous-problème 2.
\begin{lstlisting}[caption={Sous-problème 2}]
    (>\coms{\{Pré $\equiv N = N_{0} \land T = T_{0} \land 0 < k \leq N - 1$\}}<)
    int i = 0;
    (>\coms{\{Inv $\equiv N = N_{0} \land T = T_{0} \land i = 0 < k \leq N - 1$\}}<)
    while(i < k && T[i] == T[N - k + i]){
      (>\coms{\{Inv $\land B \equiv N = N_{0} \land T = T_{0} \land 0 \leq i < k < N - 1 \land T[i] = T[N - k + i]$\}}<) 
      i++;
      (>\coms{\{$N = N_{0} \land T = T_{0} \land 0 < i \leq k < N - 1 \land T[i] = T[N - k + i]$\}}<)
    }
    (>\coms{\{Inv $\land \lnot B \equiv T = T_{0} \land i == k \lor T[i] \ne T[N - k + i] \land i = \text{max\_prefixe\_suffixe}(*T, i, N)$\}}<)
\end{lstlisting}

