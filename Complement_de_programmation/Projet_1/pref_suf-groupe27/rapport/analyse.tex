% !TEX root = ./main.tex
%%%%%%%%%%%%%%%%%%%%%%%%%%%%%%%%%%%%%%%%%%%%%%%%%%%%%%%%%%%%%%%%%%%%%%%%%%%%%%%%%%%%%%%%%%
% Dans ce fichier, vous devez définir (Input/Output/O.U.) proprement et clairement le    %
% problème.
%
% Il est aussi demandé de réaliser une analyse complète (i.e., découpe en SPs)           %
%%%%%%%%%%%%%%%%%%%%%%%%%%%%%%%%%%%%%%%%%%%%%%%%%%%%%%%%%%%%%%%%%%%%%%%%%%%%%%%%%%%%%%%%%%

\section{Définition et Analyse du Problème}\label{analyse}
%%%%%%%%%%%%%%%%%%%%%%%%%%%%%%%%%%%%%%%%%%%%
\subsection{Définition du Problème}

\subsubsection{Input}
\begin{itemize}
   \item $T$ : Un tableau d'entiers
   \item $N$ : La taille du tableau
\end{itemize}

\subsubsection{Output}
\begin{itemize}
   \item La taille $k$ du plus long préfixe qui est aussi un suffixe du tableau
\end{itemize}

\subsubsection{Caractérisation des inputs}
\begin{itemize}
   \item \textbf{T} est un tableau non nul d'entiers
      \begin{itemize}
         \item \textbf{int} *T;
      \end{itemize}
   \item \textbf{N} est un entier non signé tel que $N \geq 0$
      \begin{itemize}
         \item \textbf{unsigned int} N;
      \end{itemize}
\end{itemize}

\subsection{Analyse du Problème}

On va découper le problème en sous-problèmes (SPs) afin de mieux l'analyser.
Un sous-problème est une partie du problème qui peut être résolu indépendamment
et qui peut être combiné avec d'autres sous-problèmes pour résoudre le problème
global. Ils peuvent être de plusieurs types:
\begin{itemize}
   \item Lecture au clavier.
   \item Affichage à l'écran
   \item Réaliser une action (Vérifier une propriété, une condition, sommer,...)
   \item \'Enumérer des valeurs et effectuer une action
\end{itemize}
Dans notre cas, nous avons affaire à deux sous-problèmes:
\begin{itemize}
   \item \textbf{SP1} : \'Enumération des tailles possibles de préfixe-suffixe.
   \item \textbf{SP2} : Comparer les valeurs de préfixe et de suffixe.
\end{itemize}