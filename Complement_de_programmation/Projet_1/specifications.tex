% !TEX root = ./main.tex
%%%%%%%%%%%%%%%%%%%%%%%%%%%%%%%%%%%%%%%%%%%%%%%%%%%%%%%%%%%%%%%%%%%%%%%%%%%%%%%%%%%%%%%%%%
% Dans cette section, spécifiez formellement chacun des sous-problèmes.                  %
%%%%%%%%%%%%%%%%%%%%%%%%%%%%%%%%%%%%%%%%%%%%%%%%%%%%%%%%%%%%%%%%%%%%%%%%%%%%%%%%%%%%%%%%%%
\section{Specifications}\label{specifications}
%%%%%%%%%%%%%%%%%%%%%%%%

%  \item \textbf{SP1} : \'Enumération des tailles possbiles de préfixe-suffixe.
%\item \textbf{SP2} : Comparer les valeurs de préfixe et de suffixe.

\subsection{Sous-problème 1}
Nous voulons écrire une boucle dans laquelle $k$ balaye toutes les valeurs
possibles pour le préfixe-suffixe. Pour ceci définissons la fonction:
\begin{itemize}
   \item \textbf{Input:}
      \begin{itemize}
         \item $N$, la taille du tableau
      \end{itemize}
   \item \textbf{Output:}
      \begin{itemize}
         \item $k$, la taille du préfixe-suffixe
      \end{itemize}
   \item \textbf{Caractérisation de l'input:}
      \begin{itemize}
         \item $N$ est un entier non signé tel que $N \geq 0$
         \item \textbf{unsigned int} N
      \end{itemize}
\end{itemize}

\vspace{0.4cm}
Il nous faut donc écrire une boucle while qui va balayer toutes les valeurs
possibles de $k$ de $N-1$ à 0.
\begin{itemize}
   \item \textbf{Déclaration du compteur:}
      \begin{itemize}
         \item \textbf{unsigned int} $k = N-1$
      \end{itemize}
   \item \textbf{Nombres de tours dans la boucle:}
      \begin{itemize}
         \item $N-1$
      \end{itemize}
   \item \textbf{Gardien de boucle:}
      \begin{itemize}
         \item $k > 0$
      \end{itemize}
   \item \textbf{Corps de Boucle:}
   \begin{itemize}
      \item \textbf{Comparer les valeurs de préfixe et de suffixe:}
      \begin{itemize}
         \item SP2
      \end{itemize}
      \item Décrémenter k
   \end{itemize}
\end{itemize}

\begin{lstlisting}[language=C, caption=SP1]
   /*
    * @pre: (>$N$ initialisé $\land$ $N > 0$<)
    * @post: (>$T = T_0$ $\land$ $N = N_0$ $\land$ $k = max\_prefixe\_suffixe(*T,k,N)$ $\land$ $k \geq 0$<)
    */
   unsigned int k = N - 1;
   while (k > 0) {
      // SP2
      k--;
   }
\end{lstlisting}

\subsection{Sous-problème 2}
Nous voulons écrire une boucle dans laquelle on compare les valeurs de préfixe
et de suffixe. Pour ceci définissons la fonction:
\begin{itemize}
   \item \textbf{Input:}
      \begin{itemize}
         \item $T$, le tableau d'entiers
         \item $N$, la taille du tableau
         \item $k$, la taille du préfixe-suffixe
         \item $i$, le compteur de la boucle
      \end{itemize}
   \item \textbf{Output:}
      \begin{itemize}
         \item $k$, la taille du préfixe-suffixe
      \end{itemize}
   \item \textbf{Caractérisation des inputs:}
      \begin{itemize}
         \item $N$ est un entier non signé tel que $N \geq 0$
         \item \textbf{unsigned int} $N$
         \item $k$ est un entier non signé tel que $0 \leq k < N$
         \item \textbf{unsigned int} $k = N - 1$
         \item $i$ est un entier non signé tel que $0 \leq i < k$
         \item \textbf{unsigned int} $i = 0$
         \item $T$ est un tableau d'entiers tel que $T[0, N-1]$
         \item \textbf{int} $T[N]$
      \end{itemize}
\end{itemize}
%
%
%
%
%
%%%%%%%%%%%%%%%%%%%%%%%%%%%% A CONTINUER %%%%%%%%%%%%%%%%%%%%%%%%%%%%%%%%

\vspace{0.4cm}
Il nous faut donc écrire une boucle while qui va balayer toutes les valeurs
possibles de $i$ de 0 à k-1 et trouver des correspondances entre les
valeurs de préfixe $i$ et de suffixe $N-k+i$.
\begin{itemize}
   \item \textbf{Déclaration du compteur:}
      \begin{itemize}
         \item \textbf{unsigned int} $i = 0$
      \end{itemize}
   \item \textbf{Nombres de tours dans la boucle:}
      \begin{itemize}
         \item $k$
      \end{itemize}
   \item \textbf{Gardien de boucle:}
   \item \textbf{Condition de sortie:}
      \begin{itemize}
         \item $i < k $\space$ \&\& $\space$ T[i] == T[N - k +i]$
      \end{itemize}
   \item \textbf{Corps de Boucle:}
   \begin{itemize}
      \item \textbf{Vérifier si $i$ et $k$ sont égaux:}
      \begin{itemize}
      \item if (i == k)
      \end{itemize}
      \item Incrémenter i
   \end{itemize}
\end{itemize}

\vspace{0.4cm}
\begin{lstlisting}[language=C, caption=SP2]
   /*
    * @pre: (>$T$ initialisé $\land$ $N > 0$ $\land$ $k$ initialisé $\land$ $0 \leq i < k$<)
    * @post: (>$T = T_0$ $\land$ $N = N_0$ $\land$ $k = max\_prefixe\_suffixe(*T,k,N)$ $\land$ $k \geq 0$<)
    */
   unsigned int i = 0;
   while (i < k && T[i] == T[N - k + i]) {
      i++;
   }
   if (i == k) {
      return k;
   } else {
      i = 0;
   }
\end{lstlisting}