% !TEX root = ./main.tex
%%%%%%%%%%%%%%%%%%%%%%%%%%%%%%%%%%%%%%%%%%%%%%%%%%%%%%%%%%%%%%%%%%%%%%%%%%%%%%%%%%%%%%%%%%
% Dans cette section, introduisez toutes les notations mathématiques que vous jugez      %
% utiles à la réalisation du projet.                                                     %
%%%%%%%%%%%%%%%%%%%%%%%%%%%%%%%%%%%%%%%%%%%%%%%%%%%%%%%%%%%%%%%%%%%%%%%%%%%%%%%%%%%%%%%%%%
\section{Formalisation du Problème}\label{formalisation}
%%%%%%%%%%%%%%%%%%%%%%%%%%%%%%%%%%%

Soit \textbf{prefixe\_suffixe(*T, N)} , une notation telle que:
\begin{itemize}
   \item $T$ un tableau d'entiers.
   \item $N$ est la taille du tableau $(N \geq 0)$.
\end{itemize}

\vspace{0.2cm}
On a $prefixe\_suffixe (*T, N) \equiv 0 \leq k \leq N-1, T[0, k] == T[N-k, N-1]$.

\vspace{0.4cm}
Soit $max\_prefixe\_suffixe(*T, i , N)$ une notation pour le plus long 
préfixe-suffixe de T.

\vspace{0.2cm}
$max\_prefixe\_suffixe(*T, i , N) \equiv 0 \leq i \leq k < N, max(T[0, i-1] == T[N - k + i
, N-1])$.
