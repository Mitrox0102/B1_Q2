% !TEX root = ./main.tex
%%%%%%%%%%%%%%%%%%%%%%%%%%%%%%%%%%%%%%%%%%%%%%%%%%%%%%%%%%%%%%%%%%%%%%%%%%%%%%%%%%%%%%%%%%
% Dans cette section, introduisez toutes les notations mathématiques que vous jugez      %
% utiles à la réalisation du projet.                                                     %
%%%%%%%%%%%%%%%%%%%%%%%%%%%%%%%%%%%%%%%%%%%%%%%%%%%%%%%%%%%%%%%%%%%%%%%%%%%%%%%%%%%%%%%%%%
\section{Formalisation du Problème}\label{formalisation}
%%%%%%%%%%%%%%%%%%%%%%%%%%%%%%%%%%%

Soit \textbf{prefixe\_suffixe(*T, N)} , une notation telle que:
\begin{itemize}
   \item $T$ un tableau d'entiers.
   \item $N$ est la taille du tableau $(N \geq 0)$.
\end{itemize}

\vspace{0.2cm}
On a $prefixe\_suffixe (*T, N) \equiv \forall k, 0 \leq k \leq N-1 : T[0, k] = T[N-k, N-1]$.

\vspace{0.4cm}
Soit \textbf{est\_pref\_suff(T, k, N)} une notation telle que:
\begin{itemize}
   \item $T$ est un tableau d'entiers
   \item $k$ est la taille du préfixe/suffixe à tester
   \item $N$ est la taille du tableau
\end{itemize}

\vspace{0.2cm}
$est\_pref\_suff(T, k, N) \equiv \forall i, 0 \leq i < k : T[i] = T[N-k+i]$

\vspace{0.4cm}
Soit \textbf{max\_prefixe\_suffixe(*T, i , N)} une notation pour le plus long 
préfixe-suffixe de T telle que:
\begin{itemize}
   \item $T$ est un tableau d'entiers
   \item $i$ est la position de comparaison
   \item $N$ est la taille du tableau
\end{itemize}

\vspace{0.2cm}
$max\_prefixe\_suffixe(*T, i , N) \equiv \exists k, 0 \leq i \leq k < N : est\_pref\_suff(T, k, N)$
