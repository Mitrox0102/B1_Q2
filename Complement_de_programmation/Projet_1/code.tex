% !TEX root = ./main.tex
%%%%%%%%%%%%%%%%%%%%%%%%%%%%%%%%%%%%%%%%%%%%%%%%%%%%%%%%%%%%%%%%%%%%%%%%%%%%%%%%%%%%%%%%%%
% Dans cette section, indiquez le code complet (sans assertions intermédiaires) de votre %
% solution                                                                               %
%%%%%%%%%%%%%%%%%%%%%%%%%%%%%%%%%%%%%%%%%%%%%%%%%%%%%%%%%%%%%%%%%%%%%%%%%%%%%%%%%%%%%%%%%%
\section{Code Complet}\label{code}
%%%%%%%%%%%%%%%%%%%%%%%
\subsection{Code du header}
\begin{lstlisting}[caption={prefixe\_suffixe.h}]
  #ifndef __PREFIXE_SUFFIXE__
  #define __PREFIXE_SUFFIXE__
 
  /*
   * @pre: T initialisé (>$\wedge$<) N >= 0
   * @post: (>$T = T_0$ $\wedge$ $N = N_0$ $\wedge$ $k = max\_prefixe\_suffixe(*T, i , N)$<)
   */
 int prefixe_suffixe(int *T, const unsigned int N);
 
 #endif
\end{lstlisting}

\subsection{Code du module}
\begin{lstlisting}[caption={prefixe\_suffixe.c}]
    #include <assert.h>
    #include <stdlib.h>
    
    #include "prefixe_suffixe.h"
    
    
    int prefixe_suffixe(int *T, const unsigned N){
      assert(T != NULL && N >= 0);

      int k = N-1;
      int i = 0;
    
      while (k > 0){
        while (i < k && T[i] == T[N - k +i]) {
          i++;
        } // fin while
        if (i == k) {
          return k;
        }
        else{
          i = 0;
        }
        k--;
      } // fin while
      return 0;
    } // fin prefixe\_suffixe
\end{lstlisting}

\subsection{Code du programme principal}
\begin{lstlisting}[caption={main-prefixe\_suffixe.c}]
    #include <stdio.h>

    #include "prefixe_suffixe.h"
    
    #define N1 9
    #define N2 10
    #define N3 9
    
    int main(){
    
      int T1[N1] = {1,4,2,4,5,1,4,2,4};
      int T2[N2] = {1,2,3,2,1,1,2,3,2,1};
      int T3[N3] = {3,2,3,2,1,2,3,2,1};
      int T4[N1] = {1,1,1,1,1,1,1,1,1};
    
      printf("Longueur plus long préfixe/suffixe de T1: %u\n", 
            prefixe_suffixe(T1, N1));
      printf("Longueur plus long préfixe/suffixe de T2: %u\n", 
            prefixe_suffixe(T2, N2));
      printf("Longueur plus long préfixe/suffixe de T3: %u\n", 
            prefixe_suffixe(T3, N3));
      printf("Longueur plus long préfixe/suffixe de T4: %u\n", 
            prefixe_suffixe(T4, N1));
    }
\end{lstlisting}


