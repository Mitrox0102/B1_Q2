% !TEX root = ./main.tex
%%%%%%%%%%%%%%%%%%%%%%%%%%%%%%%%%%%%%%%%%%%%%%%%%%%%%%%%%%%%%%%%%%%%%%%%%%%%%%%%%%%%%%%%%%
% Dans cette section, vous devez étudier complètement la complexité de votre code.       %
% Soyez le plus formel (i.e., mathématique) possible.                                    %
%%%%%%%%%%%%%%%%%%%%%%%%%%%%%%%%%%%%%%%%%%%%%%%%%%%%%%%%%%%%%%%%%%%%%%%%%%%%%%%%%%%%%%%%%%
\section{Complexité}\label{complexite}
%%%%%%%%%%%%%%%%%%%%

Décomposons le code de la fonction prefixe\_suffixe en plusieurs blocs :
\begin{lstlisting}[caption={T(A)}]
   int k = N-1;
   int i = 0;
\end{lstlisting}

\begin{lstlisting}[caption={T(B1)}]
    while (k > 0){
      // T(B2)
      // T(C)
      // T(D)
   }
\end{lstlisting}

\begin{lstlisting}[caption={T(B2) et T(B2')}]
   while(i < k && T[i] == T[N - k +i]){
      i++; // T(B2')
   }
\end{lstlisting}

\begin{lstlisting}[caption={T(C)}]
   if (i == k){
      return k;
   }
\end{lstlisting}

\begin{lstlisting}[caption={T(D)}]
   else{
      i = 0;
   }
\end{lstlisting}

\begin{lstlisting}[caption={T(E)}]
    k--;
\end{lstlisting}

Dans le pire des cas, on entre dans la première boucle N - 1 fois 
(lorsque $k$ va de $N - 1$ à $1$).
Dans la boucle imbriquée, on entre jusqu'à $k$ fois. On a donc

On a donc, pour chaque valeur de $k$ :

\[
T(B1) = T(B2) + max(T(C), T(D)) + T(E) = k + 2
\]

On a aussi que :

\[
T(N) = T(A) + T(B1)
\]

La complexité totale est donc :

\[
T(N) = T(A) + \sum_{k = 1}^{N-1} (k + 2) = 1 + \sum_{k = 1}^{N-1} k + \sum_{k = 1}^{N-1} 2
\]

\[
T(N) = 1 + \frac{(N - 1) \cdot N}{2} + 2(N - 1)
\]

\[
T(N) = \frac{2 + N^{2} - N + N - 1}{2}
\]

\[
T(N) = \frac{N^{2} + 1}{2}
\]

Cette fonction peut donc être bornée par $O(N^2)$.
