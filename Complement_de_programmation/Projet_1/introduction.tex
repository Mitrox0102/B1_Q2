% !TEX root = ./main.tex
%%%%%%%%%%%%%%%%%%%%%%%%%%%%%%%%%%%%%%%%%%%%%%%%%%%%%%%%%%%%%%%%%%%%%%%%%%%%%%%%%%%%%%%%%%
% Rédigez ici l'introduction de votre rapport.                                           %
%%%%%%%%%%%%%%%%%%%%%%%%%%%%%%%%%%%%%%%%%%%%%%%%%%%%%%%%%%%%%%%%%%%%%%%%%%%%%%%%%%%%%%%%%%
\section{Introduction}\label{introduction}
\subsection{Contexte}
Nous voulons construire une fonction dans laquelle nous allons passer en argument
un tableau de nombres entiers et un nombre entier positif qui représente la taille du tableau.
Nous voudrions que cette fonction nous retourne la taille du nombre d'éléments
étant à la fois préfixe et suffixe de ce tableau.

\subsection{Fonctionnement de la fonction}
Soit un tableau $T$, contenant N valeur valeurs entières $(N \geq 0)$. Nous 
voulons construire une fonction qui va retourner un entier $k$ $(k \in [0, ...,
 N-1])$ tel que le sous tableau $T[0,\space k-1]$ est un préfixe de $T$ et 
$T[N-k, N-1]$ est un suffixe de $T$, c'est-à-dire que leurs éléments soient
identiques.

Extrait du prototype de la fonction:
\begin{lstlisting}[language=C, caption=Fonction souhaitée]
int prefixe_suffixe(int *T, const unsigned int N);
\end{lstlisting}
où $T$ et $N$ ne sont pas modifiés.

\subsection{Exemple d'utilisation}
Soit F un tableau de 10 entiers et lg contient la taille du prefixe-suffixe:
\begin{lstlisting}[language=C, caption=Exemple d'utilisation]
int F[10] = {1, 2, 3, 1, 2, 3, 1, 2, 3, 1};
unsigned int lg = prefixe_suffixe(F, 10);
printf("Le plus long prefixe-suffixe du tabLeau est de taille %u.\n", lg);
\end{lstlisting}
%%%%%%%%%%%%%%%%%%%%%%%
