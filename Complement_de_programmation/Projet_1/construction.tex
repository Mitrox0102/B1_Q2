% !TEX root = ./main.tex
%%%%%%%%%%%%%%%%%%%%%%%%%%%%%%%%%%%%%%%%%%%%%%%%%%%%%%%%%%%%%%%%%%%%%%%%%%%%%%%%%%%%%%%%%%
% Dans cette section, il est demandé d'appliquer l'approche constructive pour la         %
% construction de votre code.                                                            %
%                                                                                        %
% Pour chaque Sous-Problème (une sous-section/SP):                                       %
% - {Pré} INIT {INV}                                                                     %
% - déterminer le Critère d'Arrêt (et donc le Gardien de Boucle)                         %
% - {INV & B} ITER {INV}                                                                 %
% - {INV & !B} END {Post}                                                                %
% - Fonction de Terminaison (pensez à justifier sur base de l'Invariant Graphique)       %
% (une sous-sous-section/tiret)                                                          %
%%%%%%%%%%%%%%%%%%%%%%%%%%%%%%%%%%%%%%%%%%%%%%%%%%%%%%%%%%%%%%%%%%%%%%%%%%%%%%%%%%%%%%%%%%
\section{Approche Constructive}

%%%%%%%%%%%%%%%%%%%%%%%%%%%%%%%%

\subsection{Sous-problème 1}
\begin{lstlisting}[caption={Sous-problème 1}]
    (>\coms{\{Pré $\equiv T = T_{0} \land N \geq 0$\}}<)
    unsigned int k = N - 1;
    (>\coms{\{$T = T_{0} \land N \geq 0 \land k = N-1$\}}<)
    while (k > 0) {
        // SP2
        k --;
    }
    (>\coms{\{Post $\equiv T = T_{0} \land N \geq 0$\}}<)
\end{lstlisting}


\subsection{Sous-problème 2}


\begin{lstlisting}[caption={Sous-problème 2}]
(>\coms{\{Pré $\equiv T = T_{0}$\}}<)
unsigned int k = N - 1;
(>\coms{\{$T = T_{0} \land k = N - 1$\}}<)
while(i < k && T[i] == T[N - k + i]){
    (>\coms{\{$T = T_{0} \land k = N - 1 \land i < k \land T[i] == T[N - k + i]$\}}<)
    i++;
    (>\coms{\{$T = T_{0} \land k = N - 1 \land i + 1 \leq k \land T[i] == T[N - k + i]$\}}<)
}
if (i == k){
    (>\coms{\{$T = T_{0} \land k = N - 1 \land i + 1 = k \land T[i] == T[N - k + i]$\}}<)
    return k;
}
else{
    (>\coms{\{$T = T_{0} \land k \ne N - 1 \land i < k \land T[i] == T[N - k + i]$\}}<)
    i = 0;
    (>\coms{\{$T = T_{0} \land k = N - 1 \land i = 0$\}}<)
}
(>\coms{\{Post $\equiv T = T_{0} \land i == k \lor T[i] \ne T[N - k + i]$\}}<)
\end{lstlisting}
