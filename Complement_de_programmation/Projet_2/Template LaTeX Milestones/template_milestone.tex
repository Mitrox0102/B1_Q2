%%%%%%%%%%%%%%%%%%%%%%%%%%%%%%%%%%%%%%%%%%%%%%%%%%%%%%%%%%%%%%%%%%%%%%%%%%%%%%%%%%%%%%%%%%
% Ceci est le template à utiliser pour les milestones des projets d'INFO0947.            %
%                                                                                        %
% Vous devez décommenter et compléter les commandes introduites plus bas (intitule, ...) %
% avant de pouvoir compiler le fichier LaTeX.                                            %
%                                                                                        %
% Le PDF produit doit avoir la forme suivante: grpID_ProjetX_MilestoneY.pdf              %
% où:                                                                                    %
%   - grpID est votre identifiant de groupe (cfr. eCampus)                               %
%   - X est le numéro du projet (1 pour Construction, 2 pour TAD)                        %
%   - Y est le numéro du milestone (1 ou 2)                                              %
%%%%%%%%%%%%%%%%%%%%%%%%%%%%%%%%%%%%%%%%%%%%%%%%%%%%%%%%%%%%%%%%%%%%%%%%%%%%%%%%%%%%%%%%%%

% !TEX root = ./main.tex
% !TEX engine = latexmk -pdf
% !TEX buildOnSave = true
\documentclass[a4paper, 11pt, oneside]{article}

\usepackage[utf8]{inputenc}
\usepackage[T1]{fontenc}
\usepackage[french]{babel}
\usepackage{array}
\usepackage{shortvrb}
\usepackage{listings}
\usepackage[fleqn]{amsmath}
\usepackage{amsfonts}
\usepackage{fullpage}
\usepackage{enumerate}
\usepackage{graphicx}             % import, scale, and rotate graphics
\usepackage{subfigure}            % group figures
\usepackage{alltt}
\usepackage{url}
\usepackage{indentfirst}
\usepackage{eurosym}
\usepackage{listings}
\usepackage{color}
\usepackage[table,xcdraw,dvipsnames]{xcolor}

%%%%%%%%%%%%%%%%% TITRE %%%%%%%%%%%%%%%%
% Complétez et décommentez les définitions de macros suivantes :
\newcommand{\intitule}{TAD \& Récursivité}
\newcommand{\GrNbr}{27}
\newcommand{\PrenomUN}{Alexandru}
\newcommand{\NomUN}{Dobre}
\newcommand{\PrenomDEUX}{Sami}
\newcommand{\NomDEUX}{Ouazouz}
\definecolor{\vert}{RGB}{0, 140, 70}

%%%%%%%% ZONE PROTÉGÉE : MODIFIEZ UNE DES DIX PROCHAINES %%%%%%%%
%%%%%%%%            LIGNES POUR PERDRE 2 PTS.            %%%%%%%%
\title{INFO0947: \intitule}
\author{Groupe \GrNbr : \PrenomUN~\textsc{\NomUN}, \PrenomDEUX~\textsc{\NomDEUX}}
\date{}
\begin{document}

\maketitle
%%%%%%%%%%%%%%%%%%%% FIN DE LA ZONE PROTÉGÉE %%%%%%%%%%%%%%%%%%%%

%%%%%%%%%%%%%%%% MILESTONE %%%%%%%%%%%%%%%
% Écrivez le contenu de votre milestone ci-dessous.

\section{Production}
%%%%%%%%%%%%%%%%%%%%%

%%%%%%%%%%%%%%%%%%%%%%%%%%%%%%%%%%%%%%%%%%%%%%%%%%%%%%%%%%%%%%%%%%%%%%%%%%%%%%%%%%%%%%%%%%
% Dans cette section, vous rédigez la production demandée pour le milestone.             %
%                                                                                        %
% Soyez le plus précis et complet possible.  Toute production ne correspondant pas à ce  %
% qui est demandé sera considérée comme ``hors sujet'' et la séance de feedback sera     %
% contreproductive.                                                                      %
%%%%%%%%%%%%%%%%%%%%%%%%%%%%%%%%%%%%%%%%%%%%%%%%%%%%%%%%%%%%%%%%%%%%%%%%%%%%%%%%%%%%%%%%%%

\subsection{Introduction}

Dans ce projet, notre objectif est de créer un programme visant à modéliser une
course de cycliste à plusieurs étapes. Cette course est structurée, divisée en
une séquence d'escales dans plusieurs villes où l'on fourni également des 
coordonnées géographiques et, éventuellement, le meilleur temps entegistré. Une 
course sera une séquance d'escales avec un point de départ et une arrivée.

\vspace{0.4cm}

Notre but est de fournir un programme implémentant ces structures en type
abstrait de données (TAD). Nous allons les définir de manière abstraite puis les
implémenter en C.

\vspace{0.4cm}

Nous avons 2 TAD à concevoir, un premier sachant gérer les escales et le second
les courses. De plus, le second devra pouvoir être récursif si possible.

\subsection{Définitions}

\subsubsection{TAD: Escale}

\begin{flushleft}
   \fbox{%
      \begin{minipage}{\textwidth}
      \textbf{Type:}

      \hspace{0.5cm}Escale

      \textbf{Utilise:}

      \hspace{0.5cm}Float, String

      \textbf{Opérations:}

      \hspace{0.5cm} \textcolor{red}{\textbf{create\_escale}}: Float $\times$ Float 
      $\times$ String 
      $\rightarrow$ Escale \hfill \textcolor{red}{Constructeur}

      \hspace{0.5cm} \textcolor{blue}{\textbf{set\_best\_time}}: Escale $\times
      $ Float
      $\rightarrow$ Escale \hfill \textcolor{blue}{Transformateur}

      \hspace{0.5cm} \textcolor{blue}{\textbf{calculate\_distance}}: Escale $\times
      $ Escale
      $\rightarrow$ Float

      \hspace{0.5cm} \textcolor{\vert}{\textbf{get\_x}}: Escale $\rightarrow$ Float
      \hfill \textcolor{\vert}{Observateur}

      \hspace{0.5cm} \textcolor{\vert}{\textbf{get\_y}}: Escale $\rightarrow$ Float

      \hspace{0.5cm} \textcolor{\vert}{\textbf{get\_name}}: Escale $\rightarrow$ String

      \hspace{0.5cm} \textcolor{\vert}{\textbf{get\_best\_time}}: Escale
      $\rightarrow$ Float

      \textbf{Préconditions:}

      \hspace{0.5cm} $\forall x \in Float, $\space$ \forall y \in Float, $\space$ 
      \forall name \in String:$ 

      \hspace{0.5cm} create\_escale(x, y, name) != NULL

      \textbf{Axiomes:}

      \hspace{0.5cm} $\forall e_1, e_2 \in$ Escale, $\forall x_1, x_2, y_1, y_2 \in$ Float 
      $\forall name_1, name_2 \in$ String:

      \hspace{0.5cm} $get\_x(create\_escale(x, y, name_1)) = x$ 

      \hspace{0.5cm} $get\_y(create\_escale(x, y, name_1)) = y$

      \hspace{0.5cm} $get\_name(create\_escale(x, y, name_1)) = name_1$

      \hspace{0.5cm} $get\_best\_time(create\_escale(x, y, name_1)) = 0$

      \hspace{0.5cm} $get\_best\_time(set\_best\_time(e_1, t)) = t$

      \hspace{0.5cm} $calculate\_distance(create\_escale(x_1, y_1, name_1), 
      create\_escale(x_2, y_2, name_2)) = x_2 - x_1 + y_2 - y_1$
      
      \end{minipage}
   }
\end{flushleft}

\subsubsection{TAD: Course}

\begin{flushleft}
   \fbox{%
      \begin{minipage}{\textwidth}
      \textbf{Type:}

      \hspace{0.5cm}Course

      \textbf{Utilise:}

      \hspace{0.5cm}Escale, Float, String, Boolean, Integer

      \textbf{Opérations:}

      \hspace{0.5cm} \textcolor{red}{\textbf{create\_course}}: Escale $\times$ Escale 
      $\rightarrow$ Course
      \hfill \textcolor{red}{Constructeur}

      \hspace{0.5cm} \textcolor{blue}{\textbf{add\_escale}}: Course $\times$
      Escale
      $\rightarrow$ Course
      \hfill \textcolor{blue}{Transformateur}

      \hspace{0.5cm} \textcolor{blue}{\textbf{del\_escale}}: Course
      $\rightarrow$ Course

      \hspace{0.5cm} \textcolor{\vert}{\textbf{is\_circuit}}: Course $\rightarrow$
      Boolean
      \hfill \textcolor{\vert}{Observateur}

      \hspace{0.5cm} \textcolor{\vert}{\textbf{escale\_nb}}: Course $\rightarrow$
      Integer

      \hspace{0.5cm} \textcolor{\vert}{\textbf{etape\_nb}}: Course $\rightarrow$
      Integer

      \hspace{0.5cm} \textcolor{\vert}{\textbf{get\_best\_time\_course}}: Course
      $\rightarrow$ Float

      \hspace{0.5cm} \textcolor{\vert}{\textbf{get\_best\_time\_escale}}: Course 
      $\times$ Escale $\rightarrow$ Float

      \textbf{Préconditions:}

      \hspace{0.5cm} $\forall e_1, e_2 \in$ Escale :
      

      \hspace{0.5cm} $create\_course(e_1, e_2)$\space$!= NULL$

      \hspace{0.5cm} $\forall e \in$ Escale, $\forall c \in$ Course:

      \hspace{0.5cm} $get\_best\_time\_escale(c, e)$\space$!= NULL$

      \hspace{0.5cm} $add\_escale(c, e)$\space$!= NULL$

      \hspace{0.5cm} $del\_escale(c)$\space$!= NULL$

      \textbf{Axiomes:}

      \hspace{0.5cm} $\forall e_1, e_2, e_3 \in$ Escale, $\forall c \in$ Course:

      \hspace{0.5cm} $is\_circuit(create\_course(e_1, e_2)) = False$

      \hspace{0.5cm} $get\_best\_time\_escale(create\_course(e_1, e_2)) = 0$

      \hspace{0.5cm} $escale\_nb(create\_course(e1, e2)) = 2$

      \hspace{0.5cm} $etape\_nb(create\_course(e1, e2)) = 1$

      \hspace{0.5cm} $escale\_nb(add\_escale(c, e_1)) = escale\_nb + 1$

      \hspace{0.5cm} $etape\_nb(add\_escale(c, e_1)) = etape\_nb + 1$

      \hspace{0.5cm} $escale\_nb(del\_escale(c, e_3)) = escale\_nb - 1$

      \hspace{0.5cm} $etape\_nb(del\_escale(c, e_3)) = etape\_nb - 1$

      \hspace{0.5cm} $get\_best\_time\_escale(add\_escale(c, e_3), get\_nom(e_3))
      = get\_best\_time\_(e_3)$ 

      \end{minipage}%
   }
\end{flushleft}

\section{Question(s)}
%%%%%%%%%%%%%%%%%%%%%%%

%%%%%%%%%%%%%%%%%%%%%%%%%%%%%%%%%%%%%%%%%%%%%%%%%%%%%%%%%%%%%%%%%%%%%%%%%%%%%%%%%%%%%%%%%%
% Dans cette section, vous pouvez rédiger les questions que vous avez sur le projet ou   %
% formuler ce qui reste incompris pour vous.                                             %
%                                                                                        %
% Nous en discuterons lors de la rencontre feedback sur votre production pour le         %
% milestone                                                                              %
%%%%%%%%%%%%%%%%%%%%%%%%%%%%%%%%%%%%%%%%%%%%%%%%%%%%%%%%%%%%%%%%%%%%%%%%%%%%%%%%%%%%%%%%%%

\end{document}
