% Synthèse enrichie du Chapitre 1 : Le traitement de l'information
\documentclass[12pt,a4paper]{article}
\usepackage[utf8]{inputenc}
\usepackage[T1]{fontenc}
\usepackage[french]{babel}
\usepackage{amsmath,amssymb}
\usepackage{graphicx}
\usepackage{tikz}
\usetikzlibrary{positioning}
\usepackage{fancyhdr}
\usepackage{geometry}
\usepackage{hyperref}
\usepackage{multicol}
\usepackage{enumitem}
\usepackage{booktabs}
\usepackage{longtable}
\usepackage{caption}

\geometry{margin=2.5cm}
\pagestyle{fancy}
\fancyhf{}
\rhead{Organisation des Ordinateurs - Chapitre 1}
\lhead{Synthèse enrichie}
\cfoot{\thepage}

\title{\Huge\textbf{Chapitre 1 - Le traitement de l'information}}
\author{Cours de Bernard Boigelot \newline Université de Liège}
\date{2025}

\begin{document}

\maketitle

\tableofcontents

\newpage

\section{Introduction}

L’objectif de ce chapitre est de poser les bases fondamentales de l’étude des ordinateurs : qu’est-ce qu’un ordinateur, que signifie « traiter de l’information », comment cette information circule, comment elle est mesurée et quantifiée, et pourquoi les représentations binaires sont omniprésentes dans les systèmes modernes.

\section{Qu’est-ce qu’un ordinateur ?}

\textbf{Définition :} Un ordinateur est une machine capable de \textbf{traiter des données en suivant un programme préétabli}. Il s’agit d’un automate sans initiative propre, capable d’exécuter des instructions qui lui sont données sous forme de programme.

\subsection{Exemples concrets d'applications}
\begin{itemize}
  \item \textbf{Ordinateur personnel :} bureautique, navigation web, jeu vidéo
  \item \textbf{Smartphones :} capteurs GPS, appareil photo, apps, réseaux
  \item \textbf{Voitures modernes :} dizaines de microprocesseurs (freinage, injection, multimédia...)
  \item \textbf{Objets connectés :} balances, caméras, assistants vocaux
\end{itemize}

\subsection{Exemple type examen}
\textit{Expliquez en quoi un smartphone est un ordinateur selon la définition vue au cours.}
\newline\textbf{Réponse :} Il traite des données (position, image, son, texte) en exécutant un programme (applications) préétabli par le développeur.

\section{Notion d'information}

\subsection{Définition physique}
L’information correspond à la \textbf{réduction de l’incertitude} qu’un observateur possède sur l’état d’un système. 

\textbf{Exemple :} Un dé dans une boîte fermée peut montrer un chiffre entre 1 et 6. Ouvrir la boîte nous donne une information nouvelle.

\subsection{Transmission par signaux}
L'information est transmise par \textbf{signaux} physiques :
\begin{itemize}
  \item \textbf{Optiques :} lumière (ex : faisceau laser dans un lecteur DVD)
  \item \textbf{Électriques :} tension sur un câble
  \item \textbf{Mécaniques :} position d’un interrupteur
\end{itemize}

\textbf{Exemple :} Un lecteur DVD lit une séquence de creux/reliefs sur la surface du disque. Ces éléments modulent un faisceau laser réfléchi, transformé ensuite en signal électrique.

\subsection{Chaîne complète de transmission enrichie}
\begin{center}
\begin{tabular}{|c|c|c|c|c|c|}
\hline
DVD & Faisceau laser & Photodiode & Signal électrique & Circuit électronique & Télévision \\
\hline
\end{tabular}
\end{center}

\subsection{Tableau récapitulatif des types de signaux}
\begin{longtable}{|l|l|l|}
\hline
\textbf{Type de signal} & \textbf{Exemple} & \textbf{Caractéristique} \\
\hline
Optique & DVD, fibre optique & Vitesse, haute fréquence \\
\hline
Électrique & Câble, processeur & Mesurable, codable en binaire \\
\hline
Mécanique & Interrupteur, clavier & Actions utilisateur \\
\hline
\end{longtable}

% Le reste du contenu suit sans changement
\subsection{Signaux continus : définition et limites}
\begin{itemize}
  \item Valeurs dans un intervalle dense ($[0,12]$ V)
  \item Sensibles au bruit, imprécisions (mesure, interférences)
  \item Utilisés en analogique, mais peu fiables en calcul
\end{itemize}

\subsection{Signaux discrets : robustesse et usage}
\begin{itemize}
  \item Ensemble fini de valeurs (ex : $\{0,1\}$)
  \item Résistent mieux au bruit
  \item Base des circuits logiques modernes
\end{itemize}

\subsection{Représentation binaire}
\begin{itemize}
  \item 0V = logique 0
  \item $V_{alimentation}$ = logique 1
\end{itemize}

\textbf{Schéma tension logique :}
\begin{center}
\begin{tikzpicture}
  \draw[->] (0,0) -- (5,0) node[right] {temps};
  \draw[->] (0,0) -- (0,2.5) node[above] {tension};
  \draw[thick] (0,0) -- (1,0) -- (1,2) -- (2,2) -- (2,0) -- (3,0) -- (3,2) -- (4,2);
  \node at (1.5,2.2) {\small 1};
  \node at (0.5,-0.3) {\small 0};
\end{tikzpicture}
\end{center}

\section{Quantification de l’information}

\subsection{Formule de base (Shannon)}
\[ I = \log_2 \left( \frac{1}{p} \right) \text{ bits} \]

\subsection{Tableau comparatif (valeurs typiques)}
\begin{center}
\begin{tabular}{|l|c|}
\hline
\textbf{Événement} & \textbf{Information (bits)} \\
\hline
Lancer de pièce & 1 \\
\hline
Lancer de dé (6 faces) & 2.58 \\
\hline
Lettre Z en français (0.12\%) & 9.7 \\
\hline
Lettre E (17\%) & 2.56 \\
\hline
\end{tabular}
\end{center}

\subsection{Exemples supplémentaires type examens}
\begin{enumerate}[label=\textbf{Exemple \arabic*.}, wide, labelwidth=!, labelindent=0pt]
  \item Une mesure incertaine sur 8 niveaux \Rightarrow{} $\log_2(8) = 3$ bits
  \item Signal binaire avec bruit (seuil mal réglé) \Rightarrow{} perte d’info
  \item Texte aléatoire avec 26 lettres équiprobables \Rightarrow{} $\log_2(26) \approx 4.7$ bits
  \item Lancer de 3 pièces \Rightarrow{} $2^3 = 8$ cas, donc 3 bits
\end{enumerate}

\subsection{Unités de stockage et confusion à éviter}
\begin{itemize}
  \item \textbf{Octet} = 8 bits
  \item \textbf{Préfixes binaires} : K = $2^{10}$, M = $2^{20}$
  \item \textbf{Piège} : Disque de 4 TB annoncé = 3.64 TB réels
\end{itemize}

\section{Flashcards clés pour mémorisation}
\begin{multicols}{2}
\begin{itemize}
  \item \textbf{Bit} : unité d’information
  \item \textbf{Binaire} : signal à 2 valeurs
  \item \textbf{Shannon} : $I = \log_2(1/p)$
  \item \textbf{Octet} : 8 bits
  \item \textbf{Signal discret} : fiable
  \item \textbf{Signal continu} : bruité
\end{itemize}
\end{multicols}

\section{Astuces et conseils}
\begin{itemize}
  \item Vérifier toujours si les signaux sont fiables ou affectés par du bruit
  \item Justifier les quantités d’information avec la formule de Shannon
  \item Se méfier des notations commerciales : 1TB \neq{} $2^{40}$
  \item En cas d'incertitude sur $p$, bien lire le contexte ou les fréquences données
\end{itemize}

\end{document}
