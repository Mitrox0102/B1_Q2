\section{Coopération au sein du groupe}\label{cooperation}

Notre binôme a fonctionné selon une répartition des tâches basée sur nos compétences respectives:

\begin{itemize}
    \item Muhammad s'est principalement occupé de l'interface graphique et 
    de l'intégration des composants (différentes structures de données), en plus 
    de l'alogrithme permettant de vérifier les alignements.
    \item Quant à Sami, il s'est concentré sur toute la logique du jeu avec la génération 
    aléatoire des boules, le chemin possible et la gestion des scores. 
\end{itemize}

Nous avons organisé notre travail avec des réunions hebdomadaires pour faire 
le point sur l'avancement du projet. 
La communication s'est faite par des échanges réguliers sur Discord et via les issues 
et les merges sur GitLab.

Les difficultés principales que nous avons rencontrées concernaient 
l'intégration de nos parties respectives et la compréhension de certaines subtilités de GTK+. 
Nous avons résolu ces problèmes par une collaboration étroite et en consultant la documentation et les exemples vus au cours.
