\section{Conclusion}\label{conclusion}

Ce projet nous a permis de développer une version fonctionnelle du jeu Five or More, en mettant en pratique les principes de programmation en C et les concepts vus en cours.

\subsection{Réalisations}

Nous avons réussi à implémenter l'ensemble des fonctionnalités requises :
\begin{itemize}
    \item Interface graphique complète avec GTK+
    \item Trois niveaux de difficulté avec des tailles de grille différentes
    \item Détection des alignements (horizontaux, verticaux et diagonaux)
    \item Calcul et sauvegarde des scores
    \item Génération aléatoire de boules de différentes couleurs
    \item Algorithme de recherche de chemin pour les déplacements de boules
\end{itemize}

L'architecture MVC (Modèle-Vue-Contrôleur) que nous avons adoptée nous a permis de maintenir une séparation claire entre les différents aspects du jeu, facilitant les modifications et l'évolution du code.

\subsection{Difficultés rencontrées}

Plusieurs défis ont émergé au cours du développement :
\begin{itemize}
    \item \textbf{Gestion de la mémoire} : Les fuites de mémoire ont été une préoccupation constante, nécessitant une attention particulière lors de l'allocation et de la libération des ressources.
    \item \textbf{Algorithme de recherche de chemin} : L'implémentation d'un algorithme efficace pour déterminer si un chemin existe entre deux cases a été complexe.
    \item \textbf{Détection des alignements} : La vérification des alignements dans toutes les directions a nécessité une logique rigoureuse pour éviter les erreurs.
    \item \textbf{Interface GTK+} : L'apprentissage de la bibliothèque GTK+ a représenté une courbe d'apprentissage importante, notamment pour la gestion des événements et la mise à jour dynamique de l'interface.
\end{itemize}