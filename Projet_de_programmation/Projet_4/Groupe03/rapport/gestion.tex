\section{Gestion du code}\label{gestion}

Dans le cadre de ce projet, nous avons utilisé \textbf{GitLab} comme système de 
gestion de versions (SCM - \textit{Source Control Management}). Cet outil nous 
a permis de collaborer efficacement à plusieurs sur le même dépôt de code, 
tout en assurant un suivi structuré des modifications apportées au fil du temps.
Voici le lien vers notre dépôt : \url{https://gitlab.uliege.be/Muhammad.Qayyum/info0030_groupe03.git}.

Nous avons adopté une organisation basée sur trois branches principales :
\begin{itemize}
  \item \textbf{main} : la branche principale du projet, contenant le code stable et validé ;
  \item \textbf{sami} : une branche dédiée au développement du binôme Sami ;
  \item \textbf{qayyum} : une branche dédiée au développement du binôme Muhammad. \\
\end{itemize}

Cette structuration nous a permis de travailler en parallèle sans interférer 
 avec le code des autres, limitant ainsi les conflits. 
 Chaque membre développait de nouvelles fonctionnalités ou corrections 
 sur sa propre branche, avant de fusionner (merge) ses changements avec 
 la branche \textbf{main} une fois les tests validés. \\

L'utilisation de GitLab nous a également permis de suivre l'historique 
des modifications, de revenir en arrière si nécessaire 
(par exemple avec \texttt{git checkout HEAD@\{1\}} ou \texttt{git reset --hard}),
 et de gérer efficacement les conflits lors des fusions. Le contrôle de version 
 s’est avéré essentiel pour assurer la fiabilité et la continuité du 
 développement tout au long du projet.

Grâce à cet outil, nous avons pu maintenir une bonne organisation du code, 
éviter les pertes de données, et faciliter le travail collaboratif, 
même en cas de modifications simultanées.

