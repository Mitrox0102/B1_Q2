
\section{Structures de données}\label{structures}

\subsection{Structures principales}

Nous avons conçu plusieurs structures de données opaques pour encapsuler les éléments du jeu. Voici leurs définitions complètes:

\begin{lstlisting}[language=C, caption=Structure Plateau\_t]
/**
 * @struct Plateau_t
 * @brief Structure principale representant le plateau de jeu
 */
typedef struct Plateau_t {
    int largeur;              // Nombre de colonnes
    int hauteur;              // Nombre de lignes
    int nb_types_symboles;    // Nombre de couleurs differentes
    int symboles_par_tour;    // Nombre de boules par tour
    int nb_cases_remplies;    // Nombre de cases occupees
    Case **cases;             // Tableau 2D des cases du plateau
    int difficulte;           // Niveau de difficulte
    Score score;              // Score actuel
    int nb_suivantes;         // Nombre de boules suivantes
    Couleur boules_suivantes[MAX_SUIVANTS];// Tableau des couleurs des boules suivantes
    GtkWidget **boutons_suivants;  // Boutons representant les boules suivantes
} Plateau;
\end{lstlisting}

\begin{lstlisting}[language=C, caption=Structure Case\_t]
/**
 * @struct Case_t
 * @brief Structure representant une case du plateau de jeu
 */
typedef struct Case_t {
    bool occupe;       // Indique si la case contient une boule
    Couleur col;       // Couleur de la boule (si presente)
    GtkWidget *image;  // Bouton GTK representant la case
} Case;
\end{lstlisting}

\begin{lstlisting}[language=C, caption=Structure Score\_t]
/**
 * @struct Score_t
 * @brief Structure representant un score dans le jeu
 */
typedef struct Score_t {
    char nom[MAX_NOM + 1];  // Nom du joueur
    int valeur;             // Valeur du score
} Score;
\end{lstlisting}

\begin{lstlisting}[language=C, caption=Structure Boule\_t]
/**
 * @struct Boule_t
 * @brief Structure representant une boule
 */
typedef struct Boule_t {
    Couleur couleur;    // Couleur de la boule
    GtkWidget *image;   // Image associee a la boule
} Boule;
\end{lstlisting}

\subsection{Structures auxiliaires}

En plus des structures principales, nous utilisons des structures auxiliaires pour faciliter certaines opérations:

\begin{lstlisting}[language=C, caption=Structure MiseAJour]
/**
 * @brief Structure utilisee pour la mise a jour des cases 
 * et la recherche de boutons
 */
typedef struct {
    int ligne_actuelle;      // Coordonnee ligne de la case recherchee
    int colonne_actuelle;    // Coordonnee colonne de la case recherchee
    Plateau *plateau;        // Plateau de jeu concerne
    GtkWidget *est_bouton;   // Bouton trouve
    GtkWidget *bouton_trouve;// Bouton trouve (alternative)
} MiseAJour;
\end{lstlisting}

\begin{lstlisting}[language=C, caption=Structure Data]
/**
 * @brief Structure utilisee pour les recherches de widgets dans l'interface
 */
typedef struct {
    GtkWidget *est_widget;   // Widget trouve
    gboolean est_trouve;     // Indique si un widget a ete trouve
} Data;
\end{lstlisting}

\begin{lstlisting}[language=C, caption=Structure InfosCallback]
   /**
    * @brief Structure utilisee pour stocker des informations dans les callbacks
    */
   typedef struct {
       int ligne_actuelle;      // Coordonnee ligne de la case courante
       int colonne_actuelle;    // Coordonnee colonne de la case courante
       Plateau *plateau;        // Plateau de jeu concerne
   } InfosCallback;
   \end{lstlisting}